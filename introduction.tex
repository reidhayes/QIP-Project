Quantum teleportation and superdense coding both presuppose that Alice and Bob share a maximally entangled qubit~\cite{barrett2004teleportation, bennett1992superdense}. 
This requirement was an an initial blow to hopes of using these protocols for communication.
Bennett admits in the 1992 paper that introduced superdense coding that

\hyphenblockcquote{UKenglish}{bennett1992superdense}{
The communication of two bits via two particles, one of which remains fixed while the other makes a round trip, is no more efficient in number of particles or number of transmissions than the obvious scheme of directly encoding each bit in one transmitted particle.
}

Bennett argues that there is still one advantage of superdense coding: the Bell states may be sent during `off-peak' hours, before the message has been prepared~\cite{bennett1992superdense}.

There is hope of doing even better.
Although entanglement is a precious resource for anyone hoping to build a quantum computer or channel, nature has an over-abundance of it. 
Any two systems that share a causal history, and are capable of interacting, almost certainly share some small degree of entanglement.
This is what gives rise to the high dimensionality that makes quantum systems notoriously difficult to simulate and what inspired pioneers like Richard Feynman to first propose quantum computing~\cite{feynman1982simulating}.
This natural entanglement could in principle be harnessed for quantum communication.

Any progress towards creating maximally entangled qubits from less idealized entangled states would be a step in this direction.
This would also open up the possibility of sending Bell states over lower fidelity channels.

In this report, we review the progress made by Bennett et al.\ to solve this problem.
They detail a procedure for converting a collection of partially entangled pairs into fewer maximally entangled qubits~\cite{bennett1996concentrating}.
This procedure uses only local operations~\cite{bennett1996concentrating}.
Therefore the two systems may remain spacelike separated for the duration of the procedure.
Furthermore, in the limit of large $n$ this procedure conserves expected entanglement entropy~\cite{bennett1996concentrating}.
This means that no entanglement goes to waste.
Furthermore, this conservation elevates entanglement entropy --- not shared Bell states --- as the fundamental resource for quantum communication; at least within the limitations of this method.

Bennett et al.'s method is not the final word in extracting useful entanglement from our messy real world.
It has shortcomings, which mean that, at least for now, cheap entanglement is still illusive. 
Bennett et al. assume that
\begin{enumerate}
	\item the initial state, from which we concentrate and distill Bell states, possesses only pairwise entanglement. That is, that there is no tripartite or higher order entanglement.
	\item all pairs are equally entangled. That is, the entanglement entropy of every entangled pair is the same. For two state systems, this is equivalent to every pair having the same Schmidt coefficients.
	\item there is detailed knowledge of the initial state, which must be used to decide which measurements and unitary operations should be applied throughout the procedure. That is, that the initial state is pure, using the Bayesian definition of probability.
\end{enumerate}

For certain physical systems these assumptions may hold. 
For example with entanglement formed by ions interacting in a highly symmetric crystal lattice, whose ground-state is well known.
It is more likely however, that this procedure would be applied to using low fidelity channels to transmit Bell states.
Since the message, not initial the character of the entangled states, is of cryptographic importance, Alice and Bob could communicate over unsecured classical channels and perform projective measurements until they were confident that their shared state satisfied all these assumptions.

We explain Bennett's et al.'s method in this report.
It consists of two stages.
In the first, partially entangled qubits are converted into a maximally entangled system in a smaller Hilbert space.
In the second, any means for creating maximally entangled systems of unconstrained dimension is used to generate Bell states.
Finally, we will explain Bennett et al.'s analysis of the efficiency of this method: demonstrating how this method approaches perfect yield as the size of the initial system increases.
%two stages: concentrating and standardizing