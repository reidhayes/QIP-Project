We have explained how, in the context of common quantum communication protocols, the need for maximally entangled qubits arises.
This requirement can be substituted by the weaker requirement of only partially entangled qubits, each with constant element entropy.
We have explained the procedure proposed by Bennett et al.\ for doing so: including its merits, limitations, implementation and asymptotic efficiency. 
This procedure can be done with only local operations by the two parties, although a classical communication channel can be used to reduce the required number of measurements by $1/2$.
It is limited to concentrating entanglement on initially pure states with only monogamous entanglement, and constant pairwise entanglement entropy.
We have described the implementation outline in Bennett et al.'s paper in new detail and clarity.
Finally we have explained how, using an information theoretic argument, Bennett et al.\ proved that this approach asymptotically conserves entanglement entropy as the initial system size increases.
In total, Bennett et at.'s work represents a significant contribution towards practical quantum communication.